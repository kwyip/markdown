\documentclass{beamer}
\usecolortheme{orchid}
\usecolortheme{whale}
\useinnertheme[shadow=true]{rounded}

% Localization
\usepackage{polyglossia}
\setmainlanguage{czech}

% Markdown setup
\usepackage[
  hybrid,
  footnotes,
  inlineFootnotes,
  fencedCode,
  citations,
  definitionLists,
]{markdown}
\markdownSetup{
  renderers = {
    emphasis = {\alert{#1}},
    headingOne = {\section{#1}},
    headingTwo = {\subsection{#1}},
    headingThree = {\frametitle{#1}},
    headingFour = {\framesubtitle{#1}},
    footnote = {\footnote[frame]{#1}},
  }
}

% Bibliography setup
\usepackage{filecontents}
\begin{filecontents}{main.bib}

@BOOK{knuth86,
  author        =   {Knuth, Donald Ervin},
  year          =   {1986},
  title         =   {The \TeX book},
  edition       =   {3},
  isbn          =   {0-201-13447-0},
  pagetotal     =   {ix, 479},
  publisher     =   {Addison-Westley},
  langid        =   {english},
  url           =   {https://mirrors.ctan.org/systems/knuth/dist/tex/texbook.tex},
  urldate       =   {2016-11-08},
}

@ARTICLE{carliste00,
  author        =   {David Carlisle},
  year          =   {2000},
  title         =   {XML\TeX},
  subtitle      =   {A non-validating (and not 100\% conforming) namespace-aware XML parser implemented in \TeX},
  journaltitle  =   {TUGboat},
  volume        =   {21},
  number        =   {3},
  urldate       =   {2016-11-08},
  url           =   {https://www.tug.org/TUGboat/tb21-3/tb68carl.pdf},
  pages         =   {193--199},
  issn          =   {0896-3207},
  langid        =   {english},
}

@INPROCEEDINGS{ford02,
  author        =   {Bryan Ford},
  year          =   {2002},
  title         =   {Packrat Parsing},
  subtitle      =   {Simple, powerful, lazy, linear time, functional pearl},
  booktitle     =   {ACM SIGPLAN Notices},
  volume        =   {37},
  number        =   {9},
  organization  =   {ACM},
  pages         =   {36--47},
  urldate       =   {2016-11-08},
  url           =   {http://bford.info/pub/lang/packrat-icfp02.pdf},
  doi           =   {10.1145/581478.581483},
  langid        =   {english},
}

@INPROCEEDINGS{ford04,
  author        =   {Bryan Ford},
  year          =   {2004},
  title         =   {Parsing expression grammars},
  subtitle      =   {A recognition-based syntactic foundation},
  booktitle     =   {ACM SIGPLAN Notices},
  volume        =   {39},
  number        =   {1},
  organization  =   {ACM},
  pages         =   {111--122},
  urldate       =   {2016-08-16},
  url           =   {https://pdos.csail.mit.edu/papers/parsing:popl04.pdf},
  doi           =   {10.1145/964001.964011},
  langid        =   {english},
}

@ONLINE{gruber04,
  author        =   {John Gruber},
  year          =   {2004},
  title         =   {Markdown},
  urldate       =   {2016-08-15},
  url           =   {https://daringfireball.net/projects/markdown/}, 
  langid        =   {english},
}

@ARTICLE{dominici14,
  author        =   {Massimiliano Dominici},
  year          =   {2014},
  title         =   {An overview of Pandoc},
  journaltitle  =   {TUGboat},
  volume        =   {35},
  number        =   {1},
  urldate       =   {2016-08-15},
  url           =   {http://tug.org/TUGboat/tb35-1/tb109dominici.pdf},
  pages         =   {44--50},
  issn          =   {0896-3207},
  langid        =   {english},
}

@ONLINE{macfarlane16-1,
  author        =   {John MacFarlane},
  year          =   {2016},
  title         =   {Pandoc},
  subtitle      =   {a universal document converter},
  urldate       =   {2016-08-15},
  url           =   {http://pandoc.org/}, 
  langid        =   {english}}

@ONLINE{macfarlane16-2,
  author        =   {John MacFarlane},
  year          =   {2016},
  title         =   {Lunamark},
  subtitle      =   {Lua library for conversion between markup formats},
  urldate       =   {2016-08-15},
  url           =   {https://github.com/jgm/lunamark}, 
  langid        =   {english}}

@ONLINE{luateam16,
  author        =   {{Lua Team}},
  year          =   {2016},
  title         =   {Lua},
  subtitle      =   {About},
  urldate       =   {2016-08-15},
  url           =   {https://www.lua.org/about.html},
  langid        =   {english}}

@ONLINE{luatex16,
  author        =   {{Lua\TeX{} Team}},
  year          =   {2016},
  title         =   {Lua\TeX},
  subtitle      =   {Welcome},
  urldate       =   {2016-08-15},
  url           =   {http://luatex.org/},
  langid        =   {english}}

@ONLINE{overleaf16,
  author        =   {{Overleaf}},
  year          =   {2016},
  title         =   {Two great examples of how to use \#markdown with @Overleaf
                     -- thanks @liantze!},
  urldate       =   {2016-08-15},
  url           =   {https://twitter.com/overleaf/status/763395560682364928},
  langid        =   {english}}

@BOOK{downey16,
  author        =   {Downey, Allen B. and Mayfield, Chris},
  year          =   {2016},
  title         =   {Think Java},
  subtitle      =   {How to Think Like a Computer Scientist},
  version       =   {6.1.0},
  pagetotal     =   {xviii, 273},
  publisher     =   {Green Tea Press},
  langid        =   {english},
  url           =   {http://thinkjava.org/},
  urldate       =   {2016-11-08},
}

@BOOK{gillespie16,
  author        =   {Gillespie, Colin and Lovelace, Robin},
  year          =   {2016},
  title         =   {Efficient R programming},
  isbn          =   {978-1-4919-5078-4},
  pagetotal     =   {204},
  publisher     =   {O'Reilly Media},
  langid        =   {english},
  url           =   {https://github.com/hadley/r4ds/},
  urldate       =   {2016-11-08},
}

@BOOK{grolemund16,
  author        =   {Grolemund, Garrett and Wickham, Hadley},
  year          =   {2016},
  title         =   {R for Data Science},
  isbn          =   {978-1-4919-1039-9},
  pagetotal     =   {518},
  publisher     =   {O'Reilly Media},
  langid        =   {english},
  url           =   {https://github.com/hadley/r4ds/},
  urldate       =   {2016-11-08},
}

@MANUAL{l3proj16,
  author        =   {{\LaTeX3 Project}},
  date          =   {2016-10-19},
  title         =   {The l3regex package},
  subtitle      =   {regular expressions in \TeX},
  urldate       =   {2016-11-08},
  url           =   {http://mirrors.ctan.org/macros/latex/contrib/l3experimental/l3regex.pdf},
  langid        =   {english}}

@ONLINE{novotny16-1,
  author        =   {Novotný, Vít},
  year          =   {2016},
  title         =   {Markdown},
  subtitle      =   {A package for converting and rendering markdown documents
                     inside \TeX{}},
  urldate       =   {2016-08-15},
  note          =   {Available from: \url{http://ctan.org/pkg/markdown},
                     \url{https://github.com/Witiko/markdown}, and
                     \url{https://gitlab.fi.muni.cz/xnovot32/markdown}},
  langid        =   {english}}

@MANUAL{novotny16-2,
  author        =   {Novotný, Vít},
  year          =   {2016},
  title         =   {A Markdown Interpreter for \TeX{}},
  url           =   {http://mirrors.ctan.org/macros/generic/markdown/markdown.pdf},
  urldate       =   {2016-08-17},
  langid        =   {english}}

@ONLINE{novotny16-3,
  author        =   {Novotný, Vít},
  year          =   {2016},
  title         =   {Added support for Pandoc-style citations},
  urldate       =   {2016-08-15},
  url           =   {https://github.com/jgm/lunamark/pull/20},
  langid        =   {english}}

@ARTICLE{fenn16,
  author        =   {Jürgen Fenn},
  year          =   {2016},
  title         =   {Neue Pakete auf CTAN},
  journaltitle  =   {Die \TeX nische Komödie},
  number        =   {3/2016},
  issn          =   {1434-5897},
  langid        =   {german},
}

\end{filecontents}
\usepackage[
  backend=biber,
  style=iso-authoryear,
  sorting=nty,
  autolang=other,
  sortlocale=auto,
]{biblatex}
\addbibresource{main.bib}

% Miscellaneous packages and other setup
\hypersetup{
  unicode = true,
  pdfencoding = true,
}
\usepackage{hologo}
\usepackage{minted}
\usemintedstyle{murphy}
\frenchspacing
\usepackage{pgffor}
\newcommand\becomes{%
  \vspace{1ex}%
  \foreach\n in {1,...,34}{%
    \textdownarrow~%
  }%
  \vspace{1ex}%
}

% Metadata
\title{
  Práce s jazykem \texorpdfstring{%
    \raisebox{-1.4mm}{\includegraphics[scale=0.16]{markdown-mark}}%
  }{Markdown} v rámci\\\TeX ových dokumentů}
\subtitle{%
  \texorpdfstring{\url{https://github.com/witiko/markdown}}{}%
}
\author{Vít Novotný}
%\def\CS{\textsc{c\lower.5ex\hbox{s}\kern-.075em}}
\institute{%
  Fakulta informatiky Masarykovy Univerzity, Brno%\\
% Valné shromáždění \CS\textsc{tug}u (\textsc{mendelu}, Brno)
}
\date{17. prosince 2016}
\begin{document}
\frame{\maketitle}
\AtBeginSection[]{\frame{\sectionpage}}

\begin{frame}{\contentsname}
  \tableofcontents
\end{frame}

\markdownBegin
# Úvod
## Proč používat odlehčené značkování

\begin{frame}

### \subsecname
#### Cožpak mi \TeX{} nepřijde dost dobrý?

  1. Vysoký poměr značkování vůči textu
    * @knuth86 je z 22\,\% značkování.
    * @downey16 je z 21\,\% značkování.
  2. Nulové odstínění uživatele od vnitřností \TeX u
    * Sázený dokument se nemusí zkompilovat.
      ```tex
      … v souboru {\tt zpropadená_podtržítka.tex} …
      ```
    * Sázený dokument může skončit v nekonečné smyčce.
      ```tex
      \def\whiletrue{\whiletrue}\whiletrue
      ```
    * Sázený dokument může přistupovat k příkazové řádce systému.
      ```tex
      \immediate\write18{sudo rm -rf /}
      ```
  3. Dlouhá křivka učení

\end{frame}
\begin{frame}

### \subsecname
#### Srovnání \LaTeX u s jazykem Markdown

\footnotesize
```tex
\section{Nadpis první úrovně}
\subsection{Nadpis druhé úrovně}
Odstavec se \emph{zvýrazněným textem}
\begin{quotation}
  Citát
\end{quotation}
\begin{verbatim}
  Ukázka zdrojového kódu
\end{verbatim}
\begin{itemize}
  \item Bod seznamu se \alert{silně zvýrazněným textem}
  \item Bod seznamu s odkazem%
    \footnote{Vizte \url{https://odkaz.cz} (Titulek)}
\end{itemize}
\begin{enumerate}
  \item Bod seznamu s \verb`ukázkou zdrojového kódu`
  \item Bod seznamu s obrázkem \includegraphics{obrázek.svg}
\end{enumerate}
```

\end{frame}
\begin{frame}

### \subsecname
#### Srovnání \LaTeX u s jazykem Markdown

```markdown
# Nadpis první úrovně
## Nadpis druhé úrovně
Odstavec se _zvýrazněným textem_

> Citát

␣␣␣␣Ukázka zdrojového kódu.

* Bod seznamu se **silně zvýrazněným textem**
* Bod seznamu s [odkazem](https://odkaz.cz/ "Titulek")

1. Bod seznamu s `ukázkou zdrojového kódu`
2. Bod seznamu s ![obrázkem](obrázek.svg "Titulek")
```

\end{frame}
\begin{frame}

### \subsecname
#### Jazyk Markdown

> The overriding design goal for Markdown’s formatting syntax is to make it
> *as readable as possible*. The idea is that a Markdown-formatted document
> should be _publishable as-is, as plain text_, without looking like it’s been
> marked up with tags or formatting instructions. While Markdown’s syntax has
> been influenced by several existing text-to-HTML filters, the single
> biggest source of inspiration for Markdown’s syntax is *the format of plain
> text email*.

\hfill --- @gruber04

  * Původní jazyk byl navržen pro přípravu HTML dokumentů.
  * Dnes existuje množství nástrojů (Pandoc, MultiMarkdown), které umožňují
    použití jazyka s jinými jazyky (\LaTeX).

\end{frame}
\begin{frame}

### \subsecname
#### Výhody jazyka Markdown

  1. Minimální poměr značkování vůči textu
    * @knuth86 a @downey16 jsou _z \textasciitilde 22\,\% značkování_.
    * @gillespie16 je z 5.5\,\% značkování.
    * @grolemund16 je z 3.8\,\% značkování.
  2. Úplné odstínění od vnitřností \TeX u …
    * Markdownový dokument se po převodu do \TeX u zkompiluje.
    * Dokument nemůže cyklit ani přistupovat k příkazové řádce.
  2. … nebo hybridní značkování
    * Markdown byl navržen jako doplněk HTML, ne jako náhrada.
    * Strukturně jednoduché sekce mohou být značkovány pouze Markdownem,
      složité pak pomocí \TeX ových maker.
  3. Krátká křivka učení

\end{frame}

## Existující softwarová řešení

\begin{frame}

### \subsecname
#### Švýcarský armádní nůž Pandoc

> If you need to convert files from one markup format into another,
> Pandoc is your swiss-army knife.

\hfill --- @macfarlane16-1

  * Nástroj pro konverze a víceformátové publikování.
  * Podporuje převod mezi desítkami značkovacích jazyků (Markdown, \LaTeX,
    HTML, XML Docbook) a výstupních formátů (ODF, OOXML, PDF).
  * Přípravu \LaTeX ových dokumentů pomocí Pandocu popisuje @dominici14\.

\end{frame}
\begin{frame}

### \subsecname
#### Cožpak mi ani Pandoc nepřijde dost dobrý?

  1. Nelze ovlivnit výstupní značkování.
    ```markdown
    # Nadpis {#nadpis}
    [Jeden](#nadpis) odkaz a [druhý](https://odkaz.cz/).
    ```
    \becomes
    ```tex
    \hypertarget{nadpis}{\section{Nadpis}\label{nadpis}}
    \protect\hyperlink{nadpis}{Jeden} odkaz a
    \href{https://odkaz.cz/}{druhý}.
    ```
  2. Nejedná se o součást \TeX ových distribucí.
    * Markdownové dokumenty nelze přímo editovat na kolaborativních
      \TeX ových platformách (Share\LaTeX, Overleaf).
    * Konstatní verze \TeX ové distribuce negarantuje stabilní výstup.

\end{frame}
\begin{frame}

### \subsecname
#### Cožpak mi ani Pandoc nepřijde dost dobrý?

  3. Částečné odstínění od \TeX u, částečně hybridní značkování
    * Vstupní dokument je po načtení heuristicky očištěn od speciálních
      znaků plain \TeX u:
      ```tex
      Toto {vše} 2^n \begin{bude} o~čištěno, ale \toto{už}
      nikoliv \begin{equation}2^n\end{equation} $2^n$.
      ```
      \becomes
      ```tex
      Toto \{vše\} 2\^{}n \textbackslash{}begin\{bude\}
      o\textasciitilde{}čištěno, ale \toto{už} nikoliv
      \begin{equation}2^n\end{equation} \(2^n\).
      ```
    * Nebezpečné vstupy jako
      ```tex
      \def\shell{18}\immediate\write\shell{sudo rm -rf /}
      ```
      jsou Pandocem ponechány beze změny.

\end{frame}

# Balíček `markdown.tex`
## Návrh parseru jazyka Markdown

\begin{frame}

### \subsecname
#### Je \TeX{} vhodným jazykem?

Existují parsery formálních jazyků napsané v \TeX u. Tyto parsery
rozpoznávají regulární [@l3proj16] a bezkontextové LL(1) jazyky
[@carliste00].  Markdown však není bezkontextový

```markdown
``Obrácené apostrofy (`) lze psát i v ukázkách kódu.``
```
a parser se v degenerovaných případech musí vracet přes celý vstup:
```markdown
[toto není dýmka](https://dymka.cz/ "Dýmka"
```
Implementace v \TeX u tedy je možná, ale díky absenci vhodných datových
struktur pro posun po vstupním řetězci bez kopírování také vysoce
neefektivní.
\end{frame}
\begin{frame}

### \subsecname
#### Co tedy použít namísto \TeX u?

> Lua is a powerful, efficient, lightweight, embeddable scripting language. It
> supports procedural programming, object-oriented programming, functional
> programming, data-driven programming, and data description. 

\hfill --- @luateam16
\vfill

> Lua\TeX{} is an extended version of pdf\TeX{} using Lua as an embedded
> scripting language.

\hfill --- @luatex16
\vfill

 * Interpretr Lua je dostupný všude tam, kde se nachází Lua\TeX.
 * Pomocí malého triku jej lze spustit i z pdf\TeX u a \Hologo{XeTeX}u.

\end{frame}
\begin{frame}

### \subsecname
#### Co tedy použít namísto \TeX u?

  * V Lua\TeX u můžeme přímo spouštět kód v jazyce Lua:
    ```tex
    1 + 2 = \directlua{ tex.sprint(1 + 2) }
    ```
  * V pdf\TeX u a \Hologo{XeTeX}u interpretr spustíme z příkazové řádky:
    ```tex
    1 + 2 = \newwrite\script
    \immediate\openout\script=script.lua
    \immediate\write\script{ print(1 + 2) }%
    \immediate\closeout\script
    \immediate\write18{texlua script.lua > output.tex}%
    \input output.tex
    ```

\end{frame}
\begin{frame}

### \subsecname
#### Knihovna Lunamark

  * Lunamark [@macfarlane16-2] je parser jazyka Markdown napsaný v jazyce Lua.
  * Jazyk je specifikován pomocí formalismu Parsing Expression
    Grammar~(PEG) (a s trochou podvádění) s využitím knihovny LPeg psané v
    jazyce C.
  * Veškeré závislosti knihovny byly buďto zakompilované do Lua\TeX u
    (LPeg, Slnunicode), nebo nepotřebné (Cosmo, Alt-getopt).
  * Knihovna byla vypuštěna pod licencí Expat (MIT).

\end{frame}
\begin{frame}

### \subsecname
#### Knihovna Lunamark

Knihovnu jsem modifikoval tak, aby generovala parsovací strom zakódovaný
pomocí \TeX ových maker namísto prezentačního značkování:

```markdown
# Nadpis
Toto je [odkaz](#odkaz).
```
\becomes
```tex
\markdownRendererHeadingOne{Nadpis}
Toto je \markdownRendererLink{odkaz}{#odkaz}{#odkaz}{}.
```

***

Zároveň jsem připravil makrobalík `markdown.tex`, který:

  * zkonvertuje dokument v jazyce Markdown pomocí Lunamarku,
  * zadefinuje \TeX ová makra Lunamarku a dokument vysází.

\end{frame}

## Použití balíku `markdown.tex` v \LaTeX u

\begin{frame}

### \subsecname
#### Odstíněný a hybridní režim

```tex
\documentclass{article}
\usepackage{markdown}
\begin{document}
\begin{markdown}
  Foo bar \TeX{} $2^n$.
\end{markdown}
\begin{markdown*}{hybrid}
  Foo bar \TeX{} $2^n$.
\end{markdown*}
\end{document}
```

\becomes\medskip

Foo bar \markdownRendererBackslash TeX\markdownRendererLeftBrace{}\markdownRendererRightBrace{} \markdownRendererDollarSign{}2\markdownRendererCircumflex{}n\markdownRendererDollarSign{}. Foo bar \TeX{} $2^n$.

\end{frame}
\begin{frame}

### \subsecname
#### Mapování tokenů jazyka Markdown na \TeX ová makra

```tex
\documentclass{article}
\usepackage{markdown}
\markdownSetup{renderers = {
  link = {#1\footnote{Vizte \url{#3} (#4)}},
}}
\begin{document}
\begin{markdown}
  Foo [bar](https://odkaz.cz/ "Titulek").
\end{markdown}
\end{document}
```

\becomes\medskip

Foo bar\footnote[frame]{Vizte \url{https://odkaz.cz/} (Titulek)}.

\end{frame}
\begin{frame}

### \subsecname
#### Rozšíření syntaxe

  * Jazyk Markdown má poměrně chudou syntax. Existuje proto množství
    syntaktických rozšíření, z nichž některá byla podporována již knihovnou
    Lunamark:
    * poznámky pod čarou,
    * definiční seznamy,
  * V rámci projektu byla přidána podpora pro následující syntaktická
    rozšíření:
    * citace,
    * alternativní syntax pro poznámky pod čarou,
    * oplocené ukázky zdrojového kódu.

\end{frame}
\begin{frame}

### \subsecname
#### Syntaktická rozšíření -- `\markdownSetup{footnotes}`

```tex
Toto je poznámka pod čarou,[^1] a tady je další.[^pozn]

[^1]: Zde je text poznámky.

[^pozn]: Toto je poznámka s několika odstavci.

    Následující odstavce jsou odsazené, aby bylo zřejmé,
že stále ještě náleží k poznámce.
```

\becomes\medskip
  
Toto je poznámka pod čarou,[^1] a tady je další.[^pozn]

[^1]: Zde je text poznámky.

[^pozn]: Toto je poznámka s několika odstavci.

    Následující odstavce jsou odsazené, aby bylo zřejmé,
že stále ještě náleží k poznámce.

\end{frame}
\begin{frame}

### \subsecname
#### Syntaktická rozšíření -- `\markdownSetup{inlineFootnotes}`

```tex
Toto je alternativní syntax poznámek pod čarou.^[Tyto
poznámky pod čarou se snáze píší, protože není třeba
vybírat identifikátor a přesouvat se s kurzorem pod
odstavec.]
```

\becomes\medskip

Toto je alternativní syntax poznámek pod čarou.^[Tyto
poznámky pod čarou se snáze píší, protože není třeba
vybírat identifikátor a přesouvat se s kurzorem pod
odstavec.]

\end{frame}
\begin{frame}

### \subsecname
#### Syntaktická rozšíření -- `\markdownSetup{definitionLists}`

```tex
První pojem
:   Definice

Druhý pojem
:   Definice se

    dvěma odstavci
```

\becomes\medskip
  
První pojem
:   Definice

Druhý pojem
:   Definice se

    dvěma odstavci

\end{frame}
\begin{frame}

### \subsecname
#### Syntaktická rozšíření -- `\markdownSetup{citations}`

```tex
Zde je citace s použitím závorek [@knuth86] a zde
je řetězec několika [viz @knuth86, s. 33--35;
a také @gruber04, kap. 1].

Zde je textová citace @knuth86 a zde je řetězec
několika @knuth86 [s. 33--35; @gruber04, kap. 1].
```

\becomes\medskip
  
Zde je citace s použitím závorek [@knuth86] a zde
je řetězec několika [viz @knuth86, s. 33--35;
a také @gruber04, kap. 1].

Zde je textová citace @knuth86 a zde je řetězec
několika @knuth86 [s. 33--35; @gruber04, kap. 1].

\end{frame}
\begin{frame}

### \subsecname
#### Syntaktická rozšíření -- `\markdownSetup{fencedCode}`

```
~~~ js
if (a > b)
  return c + 4;
else
  return d + 5;
~~~~~~
```

\becomes\medskip

~~~ js
if (a > b)
  return c + 4;
else
  return d + 5;
~~~~~~

\end{frame}

# \refname

\begin{frame}[allowframebreaks]
### \refname
\nocite{*}
\printbibliography
\end{frame}

\markdownEnd
\end{document}
